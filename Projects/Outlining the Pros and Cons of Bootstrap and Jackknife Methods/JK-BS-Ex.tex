% Options for packages loaded elsewhere
\PassOptionsToPackage{unicode}{hyperref}
\PassOptionsToPackage{hyphens}{url}
%
\documentclass[
]{article}
\usepackage{amsmath,amssymb}
\usepackage{iftex}
\ifPDFTeX
  \usepackage[T1]{fontenc}
  \usepackage[utf8]{inputenc}
  \usepackage{textcomp} % provide euro and other symbols
\else % if luatex or xetex
  \usepackage{unicode-math} % this also loads fontspec
  \defaultfontfeatures{Scale=MatchLowercase}
  \defaultfontfeatures[\rmfamily]{Ligatures=TeX,Scale=1}
\fi
\usepackage{lmodern}
\ifPDFTeX\else
  % xetex/luatex font selection
\fi
% Use upquote if available, for straight quotes in verbatim environments
\IfFileExists{upquote.sty}{\usepackage{upquote}}{}
\IfFileExists{microtype.sty}{% use microtype if available
  \usepackage[]{microtype}
  \UseMicrotypeSet[protrusion]{basicmath} % disable protrusion for tt fonts
}{}
\makeatletter
\@ifundefined{KOMAClassName}{% if non-KOMA class
  \IfFileExists{parskip.sty}{%
    \usepackage{parskip}
  }{% else
    \setlength{\parindent}{0pt}
    \setlength{\parskip}{6pt plus 2pt minus 1pt}}
}{% if KOMA class
  \KOMAoptions{parskip=half}}
\makeatother
\usepackage{xcolor}
\usepackage[margin=1in]{geometry}
\usepackage{color}
\usepackage{fancyvrb}
\newcommand{\VerbBar}{|}
\newcommand{\VERB}{\Verb[commandchars=\\\{\}]}
\DefineVerbatimEnvironment{Highlighting}{Verbatim}{commandchars=\\\{\}}
% Add ',fontsize=\small' for more characters per line
\usepackage{framed}
\definecolor{shadecolor}{RGB}{248,248,248}
\newenvironment{Shaded}{\begin{snugshade}}{\end{snugshade}}
\newcommand{\AlertTok}[1]{\textcolor[rgb]{0.94,0.16,0.16}{#1}}
\newcommand{\AnnotationTok}[1]{\textcolor[rgb]{0.56,0.35,0.01}{\textbf{\textit{#1}}}}
\newcommand{\AttributeTok}[1]{\textcolor[rgb]{0.13,0.29,0.53}{#1}}
\newcommand{\BaseNTok}[1]{\textcolor[rgb]{0.00,0.00,0.81}{#1}}
\newcommand{\BuiltInTok}[1]{#1}
\newcommand{\CharTok}[1]{\textcolor[rgb]{0.31,0.60,0.02}{#1}}
\newcommand{\CommentTok}[1]{\textcolor[rgb]{0.56,0.35,0.01}{\textit{#1}}}
\newcommand{\CommentVarTok}[1]{\textcolor[rgb]{0.56,0.35,0.01}{\textbf{\textit{#1}}}}
\newcommand{\ConstantTok}[1]{\textcolor[rgb]{0.56,0.35,0.01}{#1}}
\newcommand{\ControlFlowTok}[1]{\textcolor[rgb]{0.13,0.29,0.53}{\textbf{#1}}}
\newcommand{\DataTypeTok}[1]{\textcolor[rgb]{0.13,0.29,0.53}{#1}}
\newcommand{\DecValTok}[1]{\textcolor[rgb]{0.00,0.00,0.81}{#1}}
\newcommand{\DocumentationTok}[1]{\textcolor[rgb]{0.56,0.35,0.01}{\textbf{\textit{#1}}}}
\newcommand{\ErrorTok}[1]{\textcolor[rgb]{0.64,0.00,0.00}{\textbf{#1}}}
\newcommand{\ExtensionTok}[1]{#1}
\newcommand{\FloatTok}[1]{\textcolor[rgb]{0.00,0.00,0.81}{#1}}
\newcommand{\FunctionTok}[1]{\textcolor[rgb]{0.13,0.29,0.53}{\textbf{#1}}}
\newcommand{\ImportTok}[1]{#1}
\newcommand{\InformationTok}[1]{\textcolor[rgb]{0.56,0.35,0.01}{\textbf{\textit{#1}}}}
\newcommand{\KeywordTok}[1]{\textcolor[rgb]{0.13,0.29,0.53}{\textbf{#1}}}
\newcommand{\NormalTok}[1]{#1}
\newcommand{\OperatorTok}[1]{\textcolor[rgb]{0.81,0.36,0.00}{\textbf{#1}}}
\newcommand{\OtherTok}[1]{\textcolor[rgb]{0.56,0.35,0.01}{#1}}
\newcommand{\PreprocessorTok}[1]{\textcolor[rgb]{0.56,0.35,0.01}{\textit{#1}}}
\newcommand{\RegionMarkerTok}[1]{#1}
\newcommand{\SpecialCharTok}[1]{\textcolor[rgb]{0.81,0.36,0.00}{\textbf{#1}}}
\newcommand{\SpecialStringTok}[1]{\textcolor[rgb]{0.31,0.60,0.02}{#1}}
\newcommand{\StringTok}[1]{\textcolor[rgb]{0.31,0.60,0.02}{#1}}
\newcommand{\VariableTok}[1]{\textcolor[rgb]{0.00,0.00,0.00}{#1}}
\newcommand{\VerbatimStringTok}[1]{\textcolor[rgb]{0.31,0.60,0.02}{#1}}
\newcommand{\WarningTok}[1]{\textcolor[rgb]{0.56,0.35,0.01}{\textbf{\textit{#1}}}}
\usepackage{graphicx}
\makeatletter
\def\maxwidth{\ifdim\Gin@nat@width>\linewidth\linewidth\else\Gin@nat@width\fi}
\def\maxheight{\ifdim\Gin@nat@height>\textheight\textheight\else\Gin@nat@height\fi}
\makeatother
% Scale images if necessary, so that they will not overflow the page
% margins by default, and it is still possible to overwrite the defaults
% using explicit options in \includegraphics[width, height, ...]{}
\setkeys{Gin}{width=\maxwidth,height=\maxheight,keepaspectratio}
% Set default figure placement to htbp
\makeatletter
\def\fps@figure{htbp}
\makeatother
\setlength{\emergencystretch}{3em} % prevent overfull lines
\providecommand{\tightlist}{%
  \setlength{\itemsep}{0pt}\setlength{\parskip}{0pt}}
\setcounter{secnumdepth}{-\maxdimen} % remove section numbering
\ifLuaTeX
  \usepackage{selnolig}  % disable illegal ligatures
\fi
\usepackage{bookmark}
\IfFileExists{xurl.sty}{\usepackage{xurl}}{} % add URL line breaks if available
\urlstyle{same}
\hypersetup{
  pdftitle={JK BS Example},
  pdfauthor={Samuel Greeman},
  hidelinks,
  pdfcreator={LaTeX via pandoc}}

\title{JK BS Example}
\author{Samuel Greeman}
\date{}

\begin{document}
\maketitle

Since it is hard to manufacture examples of bootstrapping and
jackknifing, we will be following closely with a jackknifing example
from the math department at montana.edu and a bootstrapping example from
ucla.edu, both of which can be found in references. First, we will start
with the jackknife example:

\subsection{Reading in the data}\label{reading-in-the-data}

\begin{verbatim}
## [1] 1.544429
\end{verbatim}

\subsection{Jackknife the mean and bias
correction}\label{jackknife-the-mean-and-bias-correction}

\begin{Shaded}
\begin{Highlighting}[]
\FunctionTok{library}\NormalTok{(bootstrap)}
\NormalTok{mean\_j }\OtherTok{\textless{}{-}} \FunctionTok{jackknife}\NormalTok{(data.X, mean)}
\NormalTok{mean\_j}
\end{Highlighting}
\end{Shaded}

\begin{verbatim}
## $jack.se
## [1] 0.2518403
## 
## $jack.bias
## [1] 0
## 
## $jack.values
##  [1] 1.41015 1.55665 1.57215 1.50115 1.52815 1.45765 1.58815 1.46565 1.58615
## [10] 1.58065 1.58815 1.57765 1.51465 1.60715 1.50115 1.46715 1.57065 1.59315
## [19] 1.59865 1.55260 1.61555
## 
## $call
## jackknife(x = data.X, theta = mean)
\end{verbatim}

\begin{Shaded}
\begin{Highlighting}[]
\NormalTok{adj\_mean\_j }\OtherTok{=} \FunctionTok{mean}\NormalTok{(data.X) }\SpecialCharTok{{-}}\NormalTok{ mean\_j}\SpecialCharTok{$}\NormalTok{jack.bias}
\NormalTok{adj\_mean\_j}
\end{Highlighting}
\end{Shaded}

\begin{verbatim}
## [1] 1.544429
\end{verbatim}

As you can, see our boot-strapped mean is the same as our sample mean.
You can also see that there is no bias here in this data, but
jackknifing did change our values. Let's see what happens when we do the
jackknifing on the variance.

\subsection{Jackknife the variance and bias
correction}\label{jackknife-the-variance-and-bias-correction}

\begin{Shaded}
\begin{Highlighting}[]
\FunctionTok{var}\NormalTok{(data.X)}
\end{Highlighting}
\end{Shaded}

\begin{verbatim}
## [1] 1.331895
\end{verbatim}

\begin{Shaded}
\begin{Highlighting}[]
\NormalTok{var\_j }\OtherTok{\textless{}{-}} \FunctionTok{jackknife}\NormalTok{(data.X, var)}
\NormalTok{var\_j}
\end{Highlighting}
\end{Shaded}

\begin{verbatim}
## $jack.se
## [1] 0.3873414
## 
## $jack.bias
## [1] 0
## 
## $jack.values
##  [1] 1.003420 1.398693 1.385007 1.360590 1.396137 1.235530 1.359739 1.264808
##  [9] 1.363516 1.372992 1.359739 1.377598 1.382392 1.315033 1.360590 1.269982
## [17] 1.386796 1.349521 1.337006 1.400518 1.290180
## 
## $call
## jackknife(x = data.X, theta = var)
\end{verbatim}

\begin{Shaded}
\begin{Highlighting}[]
\NormalTok{adj\_var\_j }\OtherTok{=} \FunctionTok{var}\NormalTok{(data.X) }\SpecialCharTok{{-}}\NormalTok{ var\_j}\SpecialCharTok{$}\NormalTok{jack.bias}
\NormalTok{adj\_var\_j}
\end{Highlighting}
\end{Shaded}

\begin{verbatim}
## [1] 1.331895
\end{verbatim}

Once again, we see that our variance was unbiased. Again, our sample
values were altered, but the result was not. Jackknifing sometimes does
not tell you much, but that means that the estimator you are using is
good because it is unbiased.

Let's move onto the bootstrapping example, which uses a data set from
UCLA.

\subsection{Read in the data}\label{read-in-the-data}

After we read in the data, we set our function that we will use to
obtain statistics to bootstrap. We chose to use correlation as our
statistic:

\begin{Shaded}
\begin{Highlighting}[]
\NormalTok{cor\_fun }\OtherTok{\textless{}{-}} \ControlFlowTok{function}\NormalTok{(d, i)\{}
\NormalTok{    d2 }\OtherTok{\textless{}{-}}\NormalTok{ d[i,]}
    \FunctionTok{return}\NormalTok{(}\FunctionTok{cor}\NormalTok{(d2}\SpecialCharTok{$}\NormalTok{write, d2}\SpecialCharTok{$}\NormalTok{math))}
\NormalTok{\}}
\end{Highlighting}
\end{Shaded}

In this example we will use our number of samples as 500, and here, R
calculates our estimate, bias, and standard error for this bootstrap:

\subsection{Bootstrap execution}\label{bootstrap-execution}

\begin{Shaded}
\begin{Highlighting}[]
\NormalTok{boot\_results }\OtherTok{\textless{}{-}} \FunctionTok{boot}\NormalTok{(data.Y, cor\_fun, }\AttributeTok{R =} \DecValTok{500}\NormalTok{)}
\NormalTok{boot\_results}
\end{Highlighting}
\end{Shaded}

\begin{verbatim}
## 
## ORDINARY NONPARAMETRIC BOOTSTRAP
## 
## 
## Call:
## boot(data = data.Y, statistic = cor_fun, R = 500)
## 
## 
## Bootstrap Statistics :
##      original      bias    std. error
## t1* 0.6174493 0.001106575  0.03991453
\end{verbatim}

As you can see, this method gives us a very precise estimate of all of
our metrics.

\end{document}
